\documentclass{beamer}
\usepackage[utf8]{inputenc}
\usepackage{hyperref}
\usepackage{verbatim}
\usepackage{listings}

\usetheme{Warsaw}

\title{Programmer en Python}
\author{A. Miniussi}
\institute{Observatoire de la Côte d'Azur}

\begin{document}

\begin{frame}
  \titlepage
\end{frame}

\begin{frame}{Objectifs}
  Être capables de développer et de maintenir de façon collaborative une application d'analyse de données biologique Python.
  
  \begin{itemize}
  \item Introduction au langage Python.
  \item Principes de programmation.
  \item Développement collaboratif.
  \item Application à la bio\-informatique. 
  \end{itemize}
\end{frame}

\begin{frame}{Prérequis}
  Aucun prérequis en matière de programmation.
  
  \begin{itemize}
  \item Être capable de survivre sous Linux en mode ligne de commande.
  \item Savoir utiliser un éditeur de texte.
  \end{itemize}
\end{frame}

\begin{frame}{À savoir}
  \begin{itemize}
  \item Très peu de slides: Python et GIT disposent de beaucoup de ressources (pédagogiques, références) auxquelles nous renverrons autant que possible.
  \item Basé sur l'exemple et les travaux pratique: il faudra coder, écrire, collaborer.
  \item Non exhaustif.
  \end{itemize}
\end{frame}

\begin{frame}{Python Origin's}
  Crée à la fin des années 80 par {\em Guido Van Rossum pour}, pour faire autre chose, pendant ses vacances de Noël.
  \begin{itemize}
  \item Initié aux Pays-Bas dans le cadre du projet Amobea (mort en 1996)...
  \item ..puis au CNRI, puis à BeOpen, puis à la Python Software Foundation.
  \end{itemize}
\end{frame}

\begin{frame}{Des raisons du succès}
  Initié comme langage de script, Python promeut favorise avant tout la facilité et le confort d'utilisation.
  \begin{itemize}
  \item La plupart des librairies ``connues'' possède une interface Python.
  \item Il est facile de ``bricoler'' un script qui vous passe par la tête.
  \item Le langage offre {\em une} façon, facile à retenir, de faire les choses.
  \item La documentation fait très souvent référence au \href{https://en.wikipedia.org/wiki/Monty_Python\%27s_Flying_Circus}{Monty Python's Flying Circus}.
  \item ...
  \end{itemize}
\end{frame}

\begin{frame}{les inconvénients}
  Dépendent de vos attentes.
  \begin{itemize}
  \item Fait peu de de cas de la compatibilité ascendante $\rightarrow$ portabilité déplorable.
  \item Ne se considère pas responsable des performances d'exécution...
    \begin{itemize}
    \item qui sont rarement critiques,
    \item et souvent plus subtiles que prévues. 
    \end{itemize}
  \item Promeut parfois l'idée que développer est une activité facile.
  \end{itemize}
\end{frame}

\begin{frame}{Gestion de version}
  Si un code n'est pas appelé à évoluer, il ne mérite probablement pas d'être écrit.
  \begin{itemize}
  \item {\bf Versionnage}: Besoin de garder un trace de son travail.
  \item {\bf Travail collaboratif}: Besoin de travailler ensembles sans se marcher sur les pieds.
  \end{itemize}
  Les outils aident (beaucoup), mais ne dispensent pas de méthodologie.
\end{frame}

\begin{frame}{GIT}
  Un ``stupide pisteur de contenu'' initialement créé par Linus Torvalds, rendu utilisable par de fidèles disciples.\linebreak
  Peut signifier, selon l'humeur:
  \begin{itemize}
  \item Un ATL prononçable, libre, que l'on peut prononcer incorrectement ``get'', ce qui peut avoir un sens. Ou pas.
  \item En argot anglais, peut signifier odieux, stupide, méprisable, simplet... quelque chose de clairement péjoratif.
  \item {\bf{G}}lobal {\bf{I}}nformation {\bf{T}}racker, quand tout va bien.
  \item {\bf{G}}oddamn {\bf{I}}diotic {\bf{T}}ruckload of sh*t, quand tout va moins bien.
  \end{itemize}
  ``I'm an egotistical bastard, and I name all my projects after myself. First 'Linux', now 'git'.''
\end{frame}

\begin{frame}{Pourquoi GIT?}
  Parce que!
  \begin{itemize}
  \item Décentralisé, ce qui peut être déroutant pour les non-initiés.
  \item Rapide, ce qui peut être utile pour les gros projets.
  \item Assez complexe, ce qui peut être frustrant pour des configuration simple.
  \item Réputé ne pas se mettre en travers de la route.
  \item Peut s'interfacer avec d'autres systèmes.
  \end{itemize}
\end{frame}

\begin{frame}{Autour de GIT}
  GIT supporte plusieurs modèles de développement, et offre parfois des outils associés.
  \begin{itemize}
  \item Le {\em modèle dictatorial} (exemple: Linux), modèle pyramidal où chaque membre fait confiance à un nombre restreint de sous-fifres.
  \item Le modèle collaboratif, à base de ``pull requests'', mis en œuvre par GITHub notamment.
  \item Le modèle ``moi et mon laptop''.
  \end{itemize}
\end{frame}

\begin{frame}{L'interprète}
(ou {\em interpréteur} en franglais)

La {\em commande} {\bf python} permet d'interpréter du {\em code} {\bf Python}, au même titre que la commande {\bf perl} permet d'exécuter du code {\bf Perl} etc.. 
\note{Lancer la commande, regarder la version, comment sortir, trouver sur le web comment sortir).

On parle alors de mode {\em interactif}.

On peut s'en servir pour effectuer des essais, ou comme calculateur.

Ex: trouver comment lancer le mode interactif en partant de google/qwant/duckduckgo etc..  pour arriver sur la partie tutoriel Python-2.7 (\url{https://docs.python.org/2.7/tutorial/index.html}})

\note{Import this}

\end{frame}

\begin{frame}{Quelques types de base}

  Python propose les types de base usuels, même si ce n'est pas un langage de numéricien.
  \begin{itemize}
  \item Un type flottant, un type entier...
  \item ... définis comme étant le type ``naturel'' de la machine.
  \item Supporte les 4 opération de base, les exposant, le parenthésage..
  \end{itemize}
  Exercices:
  \begin{itemize}
  \item Évaluer des expressions, mélanger les types.
  \item Types complexes ?
  \item Diviser par zéro.
  \item Afficher $0.1$, l'ajouter à $0.2$
  \item Utiliser des variables à la place des littéraux
  \end{itemize}
\end{frame}

\begin{frame}[fragile]\frametitle{Chaînes de caractères}
  \begin{itemize}
  \item Séquence de caractères entres {\verb|'simple quotte'|} ou {\verb|"double quottes"|}
  \item Peuvent être affichées avec {\verb|print|}.
  \item On peut afficher de longues chaînes sur plusieurs ligne, ou mettre des retour à la ligne dans une chaîne.
  \end{itemize}
  \note{tester, observer le rôle de print}
  \note{essayer les accents}
  Exercices:
  \begin{itemize}
  \item Expliquer \verb|machin[1:3]|
  \item Utiliser les opérateurs + et * avec des chaînes et des entiers.
    \note{index négatifs}
  \end{itemize}
\end{frame}

\begin{frame}[fragile]\frametitle{Listes}
  La structure de donnée reine de Python.
  \begin{itemize}
  \item \verb|[3, 5, 7, 11, 13, 17]|
  \item \verb|['banana', 'grapefruit', 'strawberry', 'lemon']|
  \item \verb|[3, 'banana', 5, 3.14 ]|
  \item \verb|[[3, 4, 5], ['strawberry', 'lemon']]|
  \end{itemize}
  Exercice:
  \begin{itemize}
  \item ``Additionner'' deux liste
  \item Extraire des sous listes.
  \item Remplacer un élément.
  \item Remplacer une sous-liste.
  \item Obtenir la longueur d'un liste.
  \end{itemize}
\end{frame}

\begin{frame}[fragile]\frametitle{Conditions}
  Un algorithme doit faire des choix.
  
  \lstinputlisting[language=Python, firstline=2]{../examples/terre_plate1.py}

  Pour cela on a besoin d'un notion de vrai et de faux représentées en programmation par la notion d'expression booléenne.
  \begin{itemize}
    \item Une expression booléenne peut avoir deux valeur: \verb|True| et \verb|False|.
    \item Résultat d'une comparaison (\verb|>, <, ==, !=, >=, <=|)
    \item ... ou d'une conversion (nombres, liste etc..), éventuellement explicite: {\bf bool(}\verb|<expr>|{\bf{)}}.
  \end{itemize}
  Exercice: tester des expressions: entiers, réels, liste...
\end{frame}

\begin{frame}[fragile]\frametitle{Premières boucles}
  \center{\fbox{\lstinputlisting[language=Python]{../examples/nfirst.py}}}
  
  Exercices:
  \begin{itemize}
  \item Écrire les 10 premiers cubes.
  \item Écrire les premiers éléments de la suite de Fibonacci.
  \end{itemize}
\end{frame}

\begin{frame}[fragile]\frametitle{Premier programme}
  \begin{columns}
    \column{0.4\textwidth}
    Dans le fichier {\it fibo.py}:\newline
    \fbox{\lstinputlisting[language=Python]{../examples/fibo1.py}}
    \column{0.4\textwidth}
    À exécuter avec:
    \fbox{\lstinputlisting{../examples/fibo1.txt}}
  \end{columns}
\end{frame}

\begin{frame}[fragile]\frametitle{Premier programme autonome}
  On souhaite exécuter le programme directement:
  \begin{itemize}
  \item Il doit être exécutable.
  \end{itemize}
  \fbox{\lstinputlisting{../examples/fibo2_1.txt}}
\end{frame}

\begin{frame}[fragile]\frametitle{Premier programme autonome}
  Il faut également préciser comment il doit être exécuté:
  \fbox{\lstinputlisting{../examples/fibo2_2.txt}}

  On préférera la ligne \href{https://en.wikipedia.org/wiki/Shebang_(Unix)}{shebang} suivante:
\begin{verbatim}
!#/usr/bin/env python
\end{verbatim}
\end{frame}

\begin{frame}{Arguments}
  On souhaite pouvoir préciser combien de nombre de Fibonacci le programme affichera.

  Exercice: Retrouver la documentation concernant l'invocation de l'interprète, un indice s'y trouve peut-être.
  
  Exercice: Modifier le programme pour que l'on puisse lui transmettre ce nombre. 
\end{frame}

\begin{frame}{GIT, phase d'approche}
  Une première utilisation, minimaliste, de {\bf git} consiste à s'en servir de versionnage local.
  \begin{itemize}
  \item On empile les versions, mais sans avoir à gérer une pléthore de copies.
  \item en cas de problèmes, on peut retrouver n'importe quelle version.
  \end{itemize}
  Documentations:
  \begin{itemize}
  \item Sur le web: \url{https://git-scm.com/doc}
  \item En ligne de commande: {\tt git help}
  \end{itemize}
\end{frame}

\begin{frame}{Premier dépôt}
  Exercice: créer un dépôt local pour gérer l'évolution des codes développés pendant la formation.
  \begin{itemize}
  \item {\tt git help init}
  \item {\tt git help add}
  \item {\tt git help commit}
  \item {\tt git help checkout}
  \item {\tt git help status}
  \end{itemize}
\end{frame}

\begin{frame}{Test, Arguments}
  
  Exercices:
  \begin{itemize}
  \item On souhaite rendre l'utilisation de argument obligatoire, et afficher un message explicatif si cet argument unique n'est pas fournis.
  \item ``Commiter''  la nouvelle version.
  \item Retrouver la version précédente.
  \end{itemize}
\end{frame}

\begin{frame}{Test, Arguments}
  
  Exercices:
  \begin{itemize}
  \item On souhaite rendre l'utilisation de argument obligatoire, et afficher un message explicatif si cet argument unique n'est pas fournis.
  \item ``Commiter''  la nouvelle version.
  \item Retrouver la version précédente.
  \end{itemize}
\end{frame}

\begin{frame}{Boucles, liste}
  Exercices:
  \begin{itemize}
  \item Écrire un programme qui affiche la moyenne de ses argument.
  \item Lui faire également afficher la moyenne quadratique de ses arguments.
  \end{itemize}
\end{frame}

\begin{frame}[fragile]\frametitle{Fonctions}
  Syntaxe:
  \fbox{\lstinputlisting[firstline=5,lastline=8,language=Python]{../examples/moyenne.py}}

  Exercices:
  \begin{itemize}
  \item Écrire une fonction calculant la moyenne d'un nombre arbitraire de nombres.
  \item Reprendre l'exemple précédant en utilisant cette fonction.
  \end{itemize}
\end{frame}

\begin{frame}{Fonctions}
  La fonction peut être vue comme unité de compréhension.
  
  Découper son programme en fonction permet de le diviser en unités compréhensible quand sa taille augmente.

  \begin{itemize}
  \item De l'extérieur: on doit pouvoir comprendre ce que fait la fonction sans avoir besoin de lire son code.
  \item De l'intérieur: on doit pouvoir vérifier que la fonction fait ce qu'elle est supposée faire.
  \end{itemize}
\end{frame}

\begin{frame}{Fonctions}
  {\em De l'extérieur}
  \begin{itemize}
  \item Combien de mots vous faut il pour expliquer ce que fait votre fonction ?
  \item Est-ce que votre fonction fait une chose, et une seule.
  \item Son nom signifie t-il quelque chose de clair pour les personnes qui seront appelés à l'utiliser.
  \item Avez vous besoin de dire {\em comment} elle le fait.
  \end{itemize}    
  {\em De l'intérieur:}
  \begin{itemize}
  \item Un collègue peut-il vérifier que la fonction est correcte (fait ce qu'elle prétend faire, et rien d'autre).
  \item Sa taille lui permet elle de tenir sur une fenêtre de taille moyenne ?
  \item Les noms des variables utilisés sont-ils significatifs.
  \item Est ce que je la comprendrai dans 6 mois.
  \end{itemize}
\end{frame}

\begin{frame}[fragile]\frametitle{Récurissivîté}
  Une fonction peut s'appeler elle même:
  
  \fbox{\lstinputlisting[firstline=3,lastline=7,language=Python]{../examples/facto.py}}

\end{frame}

\begin{frame}{Saisie utilisateur}
  La fonction {\tt raw\_input} permet de lire ce qui est tapé au clavier.

  Exercice:
  \begin{itemize}
  \item Écrire une fonction qui retourne {\tt True} ou {\tt False} suivant que l'utilisateur est entré {\tt oui} ou {\tt non}.
  \item Autoriser optionnellement les majuscules
  \item Accepter deux listes: les chaînes correspondant à vrai, et celle correspondant à faux.
  \item Accepter un paramètre spécifiant le nombre d'essais autorisés.
  \end{itemize}
\end{frame}

\begin{frame}{Paramètres par défaut}
  Exercice: Donner des valeurs par défaut 'raisonnables' pour tout les paramètres sauf le prompt.
\end{frame}

\begin{frame}[fragile]\frametitle{Fonction comme variable}
  Une fonction est un objet (à peu de choses près) comme un autre.
  \fbox{\lstinputlisting{../examples/fct1classe.txt}}
\end{frame}

\begin{frame}[fragile]\frametitle{Fonction comme argument}
  Exercices:
  \begin{itemize}
  \item Écrire une fonction {\bf appliquer} prenant en argument une liste et une fonction et affichant les images des éléments.
  \item Écrire une fonction {\bf image} prenant en argument une liste et une fonction et {\em retournant} la liste des images.
  \item Ecrire une fonction {\bf combiner} prenant en argument deux fonctions à un argument et retournant queslque chose de sensé.
  \end{itemize}
\end{frame}

\begin{frame}[fragile]\frametitle{Modules}
  Constats:
  \begin{itemize}
  \item Devoir retaper son code à chaque nouvelle session python est fastidieux.
  \item Devoir écrire un script sur un seul fichier également, mais ce n'est pas le plus grave.
    \begin{itemize}
    \item Cela implique d'avoir plusieurs copies des fonctions/code.
    \item Tout code doit être maintenu (ou est probablement bon à jeter). \linebreak
      $doublerightarrow$ N copie $\equiv$ N tâche de maintenance
    \end{itemize}
  \end{itemize}
\end{frame}

\begin{frame}[fragile]\frametitle{Modules}
  \begin{columns}
    \column{0.4\textwidth}
    fibo.py:
    \newline
    \fbox{\lstinputlisting[language=Python]{../examples/fac.py}}
    \column{0.4\textwidth}
    list\_util.py:
    \newline
    \fbox{\lstinputlisting[language=Python]{../examples/list_util.py}}
  \end{columns}
  \fbox{\lstinputlisting{../examples/import.log}}
\end{frame}

\begin{frame}[fragile]\frametitle{Nommage}
  
  Le nommage des entités (fonctions, variables, modules, classe..) est l'un des problèmes les plus sous-estimés par les aspirants développeurs.\linebreak
  Un ``bon'' nom:
  \begin{itemize}
  \item Un nom doit être significatif pour le lecteur (qui doit être identifié).
  \item Un nom doit être significatif dans le(s) cadre(s) dans lequel il apparaît.
  \item Pas trop verbeux pour ne pas ennuyer le lecteur.
  \item Peut s'appuyer sur des abréviations {\em si elles sont largement acceptées par la communauté concernée}.
  \item Peut s'appuyer sur des conventions.
  \end{itemize}

  {\em Apprendre à utiliser son éditeur de code est un investissement payant.}

  {\em Le temps passé pour trouver un bon nom n'est pas du tremps perdu.}

  {\em Il faut essayer d'anticiper les usage possible et/ou probables de son code.}
  
\end{frame}

\begin{frame}[fragile]\frametitle{Import}
  
  L'utilisation de modules permet de faciliter le travail de nommage.
  \begin{itemize}
  \item En aidant à identifier le contexte d'utilisation. Un nom long étant remplacé par plusieurs nom courts.
  \item En évitant lles conflit sur les noms les plus courant.
  \item En permettant de renomer une entité lors d'un changement de contexte.
  \end{itemize}
  \fbox{\lstinputlisting{../examples/importrename.log}}
  \fbox{\lstinputlisting{../examples/importall.log}}
\end{frame}

\begin{frame}[fragile]\frametitle{Modules ou commande ?}
  Parfois, la limite entre module et commande n'est pas claire: {\bf facto.py} peut être utilisé à la fois comme module et comme commande:

  Il est possible de se baser sur le {\em nom courant} du module pour distinguer un appel de commande d'une importation de module.

  \fbox{\lstinputlisting[firstline=4]{../examples/mod_name.log}}

  En terme de bonne pratique, il est possible d'utiliser cette fonctionnalité pour effectuer des tests basiques.
  
\end{frame}

\begin{frame}{Tuples}
  \begin{itemize}
  \item Un tuple est un ``paquet'' ordonné et non modifiable de valeurs (éventuellement, d'autres tuples).
    \newline\fbox{\tt t = 1, 42, 'bob', 3.14 }
  \item Se consulte comme une liste:
    \newline\fbox{\tt nom = t[2] }
  \item  Peut être utilisé pour affecter plusieurs variables:
    \newline\fbox{\tt nom, age = `Bob', 42 }
  \item Peut être utilisé pour affecter plusieurs variables:
    \newline\fbox{\tt quotient, reste = divmod(42, 7) }
  \item Peut être mis dans dans des listes:
    \newline\fbox{\tt student=[('Bob', 15), ('Alice', 16), ('Jane',15) ]}
  \item Déjà utilisés dans la fontion de fibonacci.
  \end{itemize}
\end{frame}

\begin{frame}{Fichier}
  Les fichiers peuvent se diviser en:
  \begin{itemize}
    \item fichiers textes
      \begin{itemize}
      \item Codé en ASCII: un chiffre $\equiv$ une lettre.
      \item Interprêtables sans outils particuliers.
      \item Peut efficace pour stocker des données.
      \item Habituellement organisé en lignes.
      \end{itemize}
    \item fichiers binaires
      \begin{itemize}
      \item Contient des données brutes.
      \item A généralement besoins d'outils dédier et/ou d'une description formelle.
      \item Compact et précis.
      \item Parfois dépendant de la plate-forme.
      \end{itemize}
  \end{itemize}
\end{frame}

\begin{frame}[fragile]\frametitle{fichiers textes en Python}
  Exercices:
  \begin{itemize}
  \item Trouver la section consacrées au lecture/écriture de fichiers texte dans le tutoriel python.
  \item Sachant qu'un fichier texte est vu comme une liste de ligne, écrire une programe réaffichant les lignes d'un fichier.
    \begin{itemize}
    \item Trouver le type d'une ligne (ou pourra s'aider de la fonction {\tt type}, de l'interprète {\bf python}, des fonctions {\tt readline}, et {\tt open}).
    \item Chercher le manuel de référence de la librarie standard standard python. Y retrouver la documentation du type {\bf str}.
    \item Modifier le programme pour suprimer le retour chariot.
    \item Modifier le programme pour afficher le numero de ligne lors de l'affichage (indice: fonction {\tt enumerate}).
    \end{itemize}
  \end{itemize}
\end{frame}

\begin{frame}[fragile]\frametitle{Git et les branches}
  \begin{itemize}
  \item Git permet non seulement de versionner, mais aussi de collaborer, y compris avec soi.
  \item Votre dépôt peut héberger autant de {\em versions} différentes de votre code que vous le souhaitez.
  \item Les branche vont nous permètre de:
    \begin{itemize}
    \item Faire des essais sans comprometre votre code.
    \item De travailler sur différentes fonctionnalités en parallele.
    \item De controler les integrations de ces fonctionnalités dans votre code {``principal''}.
    \end{itemize}
  \end{itemize}
\end{frame}

\begin{frame}[fragile]\frametitle{Git et les branches}
  Exercices:
  \begin{itemize}
  \item Créer un nouveau dépot avec {\tt git init}
  \item Y ajouter le script python {\bf facto.py}, l'intégrer, commiter, taper {\tt git branch -v}
  \item Créer une nouvelle branche nommée {\bf develop}, qu'observe t'on ?
  \item Ajouter un fichier {\bf README}, controlé par git, et modifier le script.
  \item Observer les différences entre les branches {\bf develop} et {\bf master}.
  \item Sur la branche {\bf master}, modifier, différement, le script {\bf facto.py}.
  \item Intégrer les modification de la branche {\bf develop} dans la branche {\bf master}.
  \item Intégrer les modification de la branche {\bf master} dans la branche {\bf develop}.
  \end{itemize}
\end{frame}

\begin{frame}[fragile]\frametitle{Git et remote}
  Git est (aussi) un système de versionnage {\em distribué}.
  \begin{itemize}
    \item De même qu'il est possible de ``synchronizer'' des branches entre elles, il est possible de synchronizer des dépôts.
    \item Même si cela n'a rien d'obligatoire, on décide souvent de donner à l'un de ses dépôt un rôle de dépôt de référence. Les dépôt sont généralement organisés en arbres de profondeur 1 dont le dépôt de référence est la racine.
    \item Sans discipline, la situation devient rapidement ingérable.
    \item Plusieurs platforme proposent des service d'hébergement de dépôt git: \url{https://github.com}, \url{https://about.gitlab.com}...
  \end{itemize}
\end{frame}
  
\begin{frame}[fragile]\frametitle{Git et remote}
  Exercices: 
  \begin{itemize}
  \item Cloner le dépôt {\bf biopyt}
  \item Utiliser la commande {\tt git remote -v} et assurez vous que vous êtes bien sur la branche {\bf master}.
  \item Dans le répertoire {\tt examples/users/users.txt}, ajouter une ligne en respectant le format utilisé.
  \item À l'aide de la commande {\tt git push ...} intégrez votre modification sur le dépôt d'origine. Attention, suivant l'ordre dans lequel vous arrivez, les chose seront plus ou moins simples (mais toujours gérable).
  \end{itemize}
\end{frame}


\end{document}
